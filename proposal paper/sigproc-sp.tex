% THIS IS SIGPROC-SP.TEX - VERSION 3.1
% WORKS WITH V3.2SP OF ACM_PROC_ARTICLE-SP.CLS
% APRIL 2009
%
% It is an example file showing how to use the 'acm_proc_article-sp.cls' V3.2SP
% LaTeX2e document class file for Conference Proceedings submissions.
% ----------------------------------------------------------------------------------------------------------------
% This .tex file (and associated .cls V3.2SP) *DOES NOT* produce:
%       1) The Permission Statement
%       2) The Conference (location) Info information
%       3) The Copyright Line with ACM data
%       4) Page numbering
% ---------------------------------------------------------------------------------------------------------------
% It is an example which *does* use the .bib file (from which the .bbl file
% is produced).
% REMEMBER HOWEVER: After having produced the .bbl file,
% and prior to final submission,
% you need to 'insert'  your .bbl file into your source .tex file so as to provide
% ONE 'self-contained' source file.
%
% Questions regarding SIGS should be sent to
% Adrienne Griscti ---> griscti@acm.org
%
% Questions/suggestions regarding the guidelines, .tex and .cls files, etc. to
% Gerald Murray ---> murray@hq.acm.org
%
% For tracking purposes - this is V3.1SP - APRIL 2009

\documentclass{acm_proc_article-sp}
\usepackage{url}

\begin{document}


\title{Female Mosquito Simulator}
%\subtitle{[Extended Abstract]
%\titlenote{A full version of this paper is available as
%\textit{Author's Guide to Preparing ACM SIG Proceedings Using
%\LaTeX$2_\epsilon$\ and BibTeX} at
%\texttt{www.acm.org/eaddress.htm}}}
%
% You need the command \numberofauthors to handle the 'placement
% and alignment' of the authors beneath the title.
%
% For aesthetic reasons, we recommend 'three authors at a time'
% i.e. three 'name/affiliation blocks' be placed beneath the title.
%
% NOTE: You are NOT restricted in how many 'rows' of
% "name/affiliations" may appear. We just ask that you restrict
% the number of 'columns' to three.
%
% Because of the available 'opening page real-estate'
% we ask you to refrain from putting more than six authors
% (two rows with three columns) beneath the article title.
% More than six makes the first-page appear very cluttered indeed.
%
% Use the \alignauthor commands to handle the names
% and affiliations for an 'aesthetic maximum' of six authors.
% Add names, affiliations, addresses for
% the seventh etc. author(s) as the argument for the
% \additionalauthors command.
% These 'additional authors' will be output/set for you
% without further effort on your part as the last section in
% the body of your article BEFORE References or any Appendices.

\numberofauthors{2} %  in this sample file, there are a *total*
% of EIGHT authors. SIX appear on the 'first-page' (for formatting
% reasons) and the remaining two appear in the \additionalauthors section.
%
\author{
% You can go ahead and credit any number of authors here,
% e.g. one 'row of three' or two rows (consisting of one row of three
% and a second row of one, two or three).
%
% The command \alignauthor (no curly braces needed) should
% precede each author name, affiliation/snail-mail address and
% e-mail address. Additionally, tag each line of
% affiliation/address with \affaddr, and tag the
% e-mail address with \email.
%
% 1st. author
\alignauthor Nhu Tran\\
		\affaddr{University of Augsburg}\\
    \email{nhu\_tran86@yahoo.com}
% 2nd. author
\alignauthor Chris Stifter\\
		\affaddr{University of Augsburg}\\
    \email{chris.stifter@gmx.de}
}
% There's nothing stopping you putting the seventh, eighth, etc.
% author on the opening page (as the 'third row') but we ask,
% for aesthetic reasons that you place these 'additional authors'
% in the \additional authors block, viz.
%\additionalauthors{Additional authors: John Smith (The Th{\o}rv{\"a}ld Group,
%email: {\texttt{jsmith@affiliation.org}}) and Julius P.~Kumquat
%(The Kumquat Consortium, email: {\texttt{jpkumquat@consortium.net}}).}
%\date{30 July 1999}
% Just remember to make sure that the TOTAL number of authors
% is the number that will appear on the first page PLUS the
% number that will appear in the \additionalauthors section.

\maketitle
\begin{abstract}
Mosquitoes are the deadliest (and most annoying) animal family in the world. They come in a huge variety with different traits. In a lot of genera, the female mosquitoes suck blood from hosts for being able to reproduce. As a result some can transmit infectious diseases. For the hunt for blood mosquitoes are equipped with specialized senses. Within the scope of an interactive simulation, we want to present the perceived environment of mosquitoes, thus understanding the perception of the world by an alien life form. The user will be able to steer the mosquito in first person view. This will be embedded in a mini game. There, the objective for the user is to find food and blood and therefore ensure the possibility for reproduction.
\end{abstract}

% A category with the (minimum) three required fields
%\category{H.4}{Information Systems Applications}{Miscellaneous}
%A category including the fourth, optional field follows...
%\category{D.2.8}{Software Engineering}{Metrics}[complexity measures, performance measures]

%\terms{Theory}

\keywords{Interactive simulation, biology, blood-sucking (no vampires)} % NOT required for Proceedings



\section{Motivation}
Usually, the only interaction one has with a mosquito is getting stung by it or squash it at the wall. But actually mosquitoes are quite intriguing. Especially the females are interesting, because only they hunt for blood. Furthermore mosquitoes belong to the most deadliest animal family of the world~\cite{billG}, because some species transmit infectious, deadly diseases (e.g. malaria, yellow fever, dengue fever) if they bite a host. They have special senses that enable them to hunt for food and especially for blood sources~\cite{wiki_mosq}. Simulating the usage of these senses allows for an interesting and astonishing experience.


\section{Concept idea}
The interactive simulation will make it possible for the user to perceive the environment like a mosquito does. Through an appropriate mapping of the mosquito's senses to the simulation the user will be enabled to use these senses in tracking down blood sources.


\section{Project requirements}

\subsection{Science}
The life of mosquitoes has four phases: egg, larva, pupa and adult. For our interactive simulation we focus on the adult phase. As mentioned earlier only female mosquitoes feed on blood. For some species this is mandatory in order to lay eggs, while for other species it increases the amount eggs. Therefore, we want this simulation to be from the perspective of a female mosquito. The objectives a female mosquito aims to fulfill, during its adult phase, are: foraging, mating, blood sucking and laying eggs. All these objectives can usually be repeated several times.

Mosquitoes are equipped with several senses dedicated to track down blood: visual, chemical and heat sensors. The visual sensors consist of two compound eyes, which are separated from another~\cite{wiki_mosq}. With these eyes they can focus on objects that contrast with the background and especially on objects that move. Focusing on such objects yields a high probability for blood, because most likely they are alive~\cite{howstuff}. The chemical sensors represent their ability to scent. Especially carbon dioxide is identified by these sensors, because it is a result of breathing and therefore often leads to blood. Furthermore, the ability to detect heat supplements the arsenal mosquitoes have for tracking down blood.

\subsection{Gamification}
The user will steer the mosquito from first person view. His objective will be to find blood and food and to lay eggs. For finding food and eggs, he has to follow a trail of odors to the source.

There will be something like a bar, displaying the amount of health (comes with food consumption) and blood the mosquito has. The activation of sensors, like heat detection, can depend on the amount of health currently available. The user will get more points the more he used senses and when he successfully sucked enough blood and reproduced, the game can start on the next level and be more difficult. This difficulty is, for example, represented by the counter-measures of hosts, e.g. there is the possibility that a human host will smash the mosquito, while it is trying to feed on his blood.

\subsection{Complexity}
As described earlier, a main objective will be to track down the source of scents, thereby arriving at a blood source. This makes it necessary to have a simple model, that controls the propagation of the gas (carbon dioxide) the host breathes out.

Furthermore, the mosquito senses will be simulated. This means, after having done the appropriate research, finding a suitable representation of the environment. It should approach the mosquito perseption as near as possible. There we must take into account the complexity as well as the effort necessary to implement it. Possible peculiarities of the mosquito's visual senses are for example: horizontal and vertical field of view, contrast and color perception, perception of shape. 

Representing heat sensors could be done by visualizing it. This seems reasonable, because people are used to such kind of heat representation via thermal imaging cameras in films.

The remaining sense, sensing odors (chemical), is more subject to discussion. While it could also be represented by visual indicators (e.g. 'smelling clouds') this has two drawbacks: First, its rather boring, if every sense is mapped to visuals. Second, it would be necessary to only show this smelling cloud in the direct vicinity of the mosquito. Otherwise the relevance of tracking down the scent gets diminished. While this solution would be acceptible, we want to propose a different solution: mapping the smelling sense to a short sound. Depending on the density of the track we could vary the volume and frequency of the chosen sound. 

\subsection{Aesthetics}
The user interface will be as simple as possible.Depending on the research we have to do in regards to the visual senses, the environment could be very simple too. If we can abstract to a dark/light perceived environment, we could save a lot of effort creating the visual assets. If we show a rather 'normal' picture (normal relative to human perception), we most likely have to do a very narrowed, controlled environment, like only one or two rooms.



\section{Concept user-experience}
During the first usage of the simulator, pop-ups can be used to convey the necessary shortcuts for steering and the knowledge about the objectives and how to reach them. As already mentioned we want a simple user interface, with bars representing health and blood levels. If the mosquito dies, it is Game Over. If it succeeds in laying eggs, the next level starts, with the hosts being more aggressive and maybe a more difficult scent track. 

Furthermore, its possible to let the user switch between him steering the mosquito and sitting back while the computer does so. We therefore could use a random walk model which is biased to positions where the mosquito senses odors.

We will decide on this after we have completed research and worked out the final concept.


\section{Timeline}
This will be subject to change. A first estimation could be:

\begin{itemize}
	\item	8.11. Research completed
	\item 12.11. Concept finalized
	\item 29.11. Assets ready, environment designed
	\item 24.12. Logic ready,
	\item 15.1	Refactoring, bugfixing
\end{itemize}








%
% The following two commands are all you need in the
% initial runs of your .tex file to
% produce the bibliography for the citations in your paper.
\bibliographystyle{abbrv}
\bibliography{sigproc}  % sigproc.bib is the name of the Bibliography in this case
% You must have a proper ".bib" file
%  and remember to run:
% latex bibtex latex latex
% to resolve all references
%
% ACM needs 'a single self-contained file'!
%
%APPENDICES are optional
%\balancecolumns

\balancecolumns
% That's all folks!
\end{document}
